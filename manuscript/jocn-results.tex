Participants generally performed well at the task with a few participants at ceiling. Overall mean task accuracy was 91\% (SD = 6.3\%). Task accuracy varied significantly with target frequency (\textit{F}(2, 88) = 58.7, \textit{p} < .001). Post hoc pairwise comparisons with Bonferroni correction applied indicates that accuracies were significantly lower for the 200 Hz target (M = 85.6\%, SD = 9.8\%) than the 130 Hz target (M = 95.2\%, SD = 4.6\%; mean difference = 9.62\%, \textit{p} < .001) and the 280 Hz target (M = 93.1\%, SD = 6.5\%; mean difference = 7.52\%, \textit{p} < .001). Accuracies were not significantly different between the 130 Hz and 280 Hz targets (mean difference = 2.09\%, \textit{p} = .36). A median split was taken to separate participants into high and low task performers (Mdn = 92.3\%) (Figure 1). An Ordinary Least Squares regression was run between years of musical experience and task accuracy. The model was weakly statistically significant \textit{R}² = .13, \textit{F}(1, 39) = 5.03, \textit{p} = .031, with years of musical experience significantly predicting scores $\beta$ = 0.54\%, SE  = 0.24, \textit{t} = 2.24, \textit{p} = .031.





\subsection*{FFR results}
A linear mixed-effects model was fit to the data using restricted maximum likelihood (REML) in R (version 2025.05.1; Bates et al., 2015, lme4; Kuznetsova et al., 2017, lmerTest; Lenth, 2025, emmeans). The model included fixed effects of tone frequency (130, 200, 280 Hz), target frequency (130, 200, 280 Hz), and task performance (high vs. low accuracy) on FFR amplitude, and a random intercept for subject, with degrees of freedom estimated via the Satterthwaite approximation. Effect sizes for post hoc tests measured using Cohen’s d are reported for the main planned comparisons and effects only (Figures 2 and 3). 

Model fit indices showed a marginal \textit{R}² = .16, representing variance explained by fixed effects, and a conditional \textit{R}² = .55, representing variance explained by both fixed and random effects (Nakagawa \& Schielzeth, 2013, performance version 0.15.0.3). The model indicated substantial between-subject variability (random intercept variance = 8.77, SD = 2.96), with residual variance of 9.89 (SD = 3.15). The intercept of the model was significant, \textit{b} = 1.90, SE = 0.90, \textit{t}(130.06) = 2.11, \textit{p} = .037, showing that the reference condition (stimulus = 130 Hz, target = 130 Hz, high performers) elicited reliably above-zero FFR amplitudes.

The model revealed no significant main effects, including the main effects of stimulus (200 Hz: \textit{b} = 0.61, SE = 0.93, \textit{t}(320) = 0.66, \textit{p} = .51; 280 Hz: \textit{b} = 0.94, SE = 0.93, \textit{t}(320) = 1.01, \textit{p} = .31), target (200 Hz: \textit{b} = -0.75, SE = 0.93, \textit{t}(320) = -0.81, \textit{p} = .42; 280 Hz: \textit{b} = -0.57, SE = 0.93, \textit{t}(320) = -0.62, \textit{p} = .54), or task performance (low vs. high: \textit{b} = -0.01, SE = 1.34, \textit{t}(130.06) = -0.01, \textit{p} = .99). All main effects explained \textit{R}²$\beta \leq$* .1\% of the variance.

The model indicated a significant interaction between tone frequency and task performance. This interaction explains 1.2\% of the total variance in FFR amplitudes (\textit{R}²$\beta$* = .012, 95\% CI [.001, .046]). FFR amplitudes were significantly higher in low task performers compared to high task performers for the higher frequency tones. When the stimulus frequency was 200 Hz, the low performers (M = 5.76 dB, 95\% CI [3.59, 7.93]) showed higher FFR amplitudes relative to the high performers (M = 2.98 dB, 95\% CI [1.01, 4.96]; difference = -2.77 dB (SE = 1.08, \textit{t}(59.4) = 2.58, \textit{p} = .013, \textit{d} = -0.88, 95\% CI [0.19, 1.57]). Similarly, when the stimulus frequency was 280 Hz, the low performers (M = 6.77 dB, 95\% CI [4.60, 8.94]) showed higher FFR amplitudes relative to the high performers (M = 3.20 dB, 95\% CI [1.23, 5.17]; difference = 3.57 dB, SE = 1.08, \textit{t}(59.4) = 3.32, \textit{p} = .002, \textit{d} = 1.14, 95\% CI [0.45, 1.83]). There was no difference in FFR amplitudes between high (M = 1.46 dB, 95\% CI [-0.51, 3.43]) and low performers (M = 3.34 dB, 95\% CI [1.17, 5.51]) to the 130 Hz tone (difference = 1.88 dB, SE = 1.08, \textit{t}(59.4) = 1.75, \textit{p} = .09, \textit{d} =0.60, 95\% CI [0.09, 1.29]).

The interaction between target frequency and task performance also reached significance. High task performers had lower FFR amplitudes across all target tones. This effect accounted for 0.8\% of the variance in FFR amplitudes (\textit{R}²$\beta$* = .008, 95\% CI [.001, .039]). When the target was 130 Hz, low performers (M = 4.63 dB, 95\% CI [2.46, 6.80]) had a higher FFR amplitude than high task performers (M = 2.42 dB, 95\% CI [0.45, 4.39]; difference = 2.22 dB, SE = 1.08, \textit{t}(59.4) = 2.06, \textit{p} = .044, \textit{d} = 0.70, 95\% CI [0.02, 1.39]). Similarly, when the target was 200 Hz, low performers (M = 5.47 dB, 95\% CI [3.30, 7.64]) had a higher FFR amplitude than high task performers (M = 2.23 dB, 95\% CI [0.26, 4.20], difference = 3.24 dB, SE = 1.08, \textit{t}(59.4) = 3.01, \textit{p} = .004, \textit{d} = 1.03, 95\% CI [0.34, 1.72]). Lastly, when the target was 280 Hz, low performers (M = 5.77, 95\% CI [3.60, 7.94]) had a higher FFR amplitude than high task performers (M = 3.00, 95\% CI [1.03, 4.97], difference = 2.77 dB, SE = 1,08, \textit{t}(59.4) = 2.58, \textit{p} = .013, \textit{d} = 0.88, 95\% CI [0.19, 1.57].

Importantly, the interaction between tone frequency and target frequency and the three-way interactions between stimulus, target, and task performance did not reach significance (both \textit{p} > .83). These interactions explained < 0.4\% of the variance. 


\subsection*{PLV results}
To verify whether the observed differences in FFR amplitude may be attributed to reduced phase locking consistency, PLV values were calculated for each tone for both low and high-performing participants. As described in the methods, PLV values were computed by bootstrapping values from samples of randomly drawn trials, while the null distribution was created by computing the mean of randomly drawn phase values 1000 times (Mdn = 0.044, IQR = 0.001, \textit{B} = 1000). Since a majority of the observed PLV values were not normally distributed, their statistical significance was calculated as the proportion of values in the null distribution greater than the median of the observed PLVs. To compare the PLVs of high and low task performers, Mann-Whitney U tests were used to compare group medians alongside Cohen’s r to measure effect size (Figure 4).

PLVs for all performance groups to all tones were significantly greater than the bootstrapped null. The median PLV to the 130 Hz tone for low performers (Mdn = 0.046, IQR = 0.005) with a BCa 95\% CI of [0.045, 0.048] was significantly greater than null (\textit{p} = .014, \textit{B} = 1000). Similarly, phase locking to the 130 Hz tone for high performers was Mdn = 0.049 (IQR = 0.005) with a BCa 95\% CI of [0.046, 0.050], which was significantly greater than null (\textit{p} < .001, \textit{B} = 1000). Phase locking to the 200 Hz tone for low performers (Mdn = 0.048, IQR = 0.011) with a BCa 95\% CI of [0.044, 0.052] and high performers (Mdn = 0.050, IQR = 0.012) with a BCa 95\% CI of [0.045, 0.057] were significantly greater than null (\textit{p} < .001, and \textit{p} <  .001 respectively, \textit{B} = 1000). Lastly, phase locking to the 280 Hz tone for low performers (Mdn = 0.051, IQR = 0.010) with a BCa 95\% CI of [0.046, 0.054] and high performers (Mdn = 0.047, IQR = 0.004) with a BCa 95\% CI of [0.045, 0.049] were significantly greater than null (\textit{p} < .001, and \textit{p} < .001 respectively, \textit{B} = 1000).

Phase locking was not significantly different between high and low task performers for the 200 and 280 Hz stimulus tones. The Mann-Whitney U tests between high and low task performers to the 200 Hz tone yielded U = 243.0, \textit{p} = 0.27 with a small effect size of r = 0.10. The test between high and low task performers to the 280 Hz tone yielded U = 177.0, \textit{p} = 0.86 with a small effect size of r = -0.16. In contrast to the higher tones, the test between PLVs for high and low task performers to the 130 Hz tone suggests lower phase locking consistency in low task performers U = 289.0, \textit{p} = 0.04 with a medium effect size of r = 0.27.

