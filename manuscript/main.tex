Participants found this task relatively easy, with a few participants were at performance ceiling. The overall mean accuracy was 91\% (SD = 6.4\%). Performance varied significantly depending on which tone served as the target (F(2, 46) = 19.23, p < 0.01). Post hoc pairwise comparisons with Bonferroni correction revealed that accuracy was significantly lower for trials with the 200 Hz tone as the target (M = 85.1\%, SD = 10.6\%) compared to both the 130 Hz target (M = 94.6\%, SD = 5.2\%; mean difference = 9.5\%, t(22) = 4.62, p < 0.01) and the 280 Hz target (M = 92.9\%, SD = 6.2\%; mean difference = 7.8\%, t (22)= 5.14, p < 0.01). There was no significant difference in accuracy between the 130 Hz and 280 Hz targets (mean difference = 1.7\%, t(22) = 1.23, p = 0.23). 

To facilitate subsequent analyses, a median split was used to divide participants into high and low task performers (Mdn = 91.3\%). Additionally, a simple linear regression was performed between years of musical practice and task accuracy, revealing a significant positive correlation  $\beta$ = 0.62\%, SE  = 0.03, \textit{t} = 2.33, \textit{p} = .029. A simple linear regression was also performed between the d-prime scores for the auditory N-back (calculated as the difference between the z-score for hits, and the z-score for false alarms), and task accuracy, although this regression did not reach significance $\beta$ = 5.92\%, SE  = 3.8, \textit{t} = 1.5, \textit{p} = .14.


\subsection*{FFR results}
Targeted mixed-effects models selected to align with each specific research questions were fit to the data. These models were chosen to avoid an over-parameterized omnibus model and provides clearer estimates for each hypothesis. All models were fit using restricted maximum likelihood (REML) in R (version 2025.05.1; Bates et al., 2015, lme4; Kuznetsova et al., 2017, lmerTest; Lenth, 2025, emmeans) with random intercepts for participants. Degrees of freedom were estimated via the Satterthwaite approximation and effect sizes for post hoc tests, measured using Cohen’s d, are reported for planned comparisons and effects only. Model fit was assessed by computing the marginal and conditional$R^2$ values and the semi-partial $R^2$ value for each predictor (Nakagawa \& Schielzeth, 2013, performance version 0.15.0.3). The first model included fixed effects for tone frequency (130, 200, 280 Hz) and listening condition (passive vs active) to address the effect of listening condition on FFR amplitudes. The second model addressed the effects of selective auditory attention to stimuli within an attended stream, with fixed effects for tone frequency (130, 200, 280 Hz) and target frequency (130, 200, 280 Hz). The last model tested the interaction between task performance and listening condition on the FFR using fixed effects for tone (130, 200, 280 Hz), condition (passive vs active) and task performance (low vs high). 

The first model with fixed effects for listening condition and tone frequency with random intercepts for participants. The fixed effect of listening condition accounted for 26\% of the variance (marginal $R^2$), and the full model including subject-level random effects explained 56\% (conditional $R^2$), reflecting strong individual-level predictability. 



The model intercept indicated strong between-subjects variability (random intercept variance = ??, SE = ??????), with residual variance of 7.07, SD = 2.66. The intercept of the model was significant TOO MANY STATS








\subsection*{FFR results}
A linear mixed-effects model was fit to the data using restricted maximum likelihood (REML) in R (version 2025.05.1; Bates et al., 2015, lme4; Kuznetsova et al., 2017, lmerTest; Lenth, 2025, emmeans). The model included fixed effects of tone frequency (130, 200, 280 Hz), target frequency (130, 200, 280 Hz), and task performance (high vs. low accuracy) on FFR amplitude, and a random intercept for subject, with degrees of freedom estimated via the Satterthwaite approximation. Effect sizes for post hoc tests measured using Cohen’s d are reported for the main planned comparisons and effects only (Figures 2 and 3). 

Model fit indices showed a marginal \textit{R}² = .16, representing variance explained by fixed effects, and a conditional \textit{R}² = .55, representing variance explained by both fixed and random effects (Nakagawa \& Schielzeth, 2013, performance version 0.15.0.3). The model indicated substantial between-subject variability (random intercept variance = 8.77, SD = 2.96), with residual variance of 9.89 (SD = 3.15). The intercept of the model was significant, \textit{b} = 1.90, SE = 0.90, \textit{t}(130.06) = 2.11, \textit{p} = .037, showing that the reference condition (stimulus = 130 Hz, target = 130 Hz, high performers) elicited reliably above-zero FFR amplitudes.

The model revealed no significant main effects, including the main effects of stimulus (200 Hz: \textit{b} = 0.61, SE = 0.93, \textit{t}(320) = 0.66, \textit{p} = .51; 280 Hz: \textit{b} = 0.94, SE = 0.93, \textit{t}(320) = 1.01, \textit{p} = .31), target (200 Hz: \textit{b} = -0.75, SE = 0.93, \textit{t}(320) = -0.81, \textit{p} = .42; 280 Hz: \textit{b} = -0.57, SE = 0.93, \textit{t}(320) = -0.62, \textit{p} = .54), or task performance (low vs. high: \textit{b} = -0.01, SE = 1.34, \textit{t}(130.06) = -0.01, \textit{p} = .99). All main effects explained \textit{R}²$\beta \leq$* .1\% of the variance.

The model indicated a significant interaction between tone frequency and task performance. This interaction explains 1.2\% of the total variance in FFR amplitudes (\textit{R}²$\beta$* = .012, 95\% CI [.001, .046]). FFR amplitudes were significantly higher in low task performers compared to high task performers for the higher frequency tones. When the stimulus frequency was 200 Hz, the low performers (M = 5.76 dB, 95\% CI [3.59, 7.93]) showed higher FFR amplitudes relative to the high performers (M = 2.98 dB, 95\% CI [1.01, 4.96]; difference = -2.77 dB (SE = 1.08, \textit{t}(59.4) = 2.58, \textit{p} = .013, \textit{d} = -0.88, 95\% CI [0.19, 1.57]). Similarly, when the stimulus frequency was 280 Hz, the low performers (M = 6.77 dB, 95\% CI [4.60, 8.94]) showed higher FFR amplitudes relative to the high performers (M = 3.20 dB, 95\% CI [1.23, 5.17]; difference = 3.57 dB, SE = 1.08, \textit{t}(59.4) = 3.32, \textit{p} = .002, \textit{d} = 1.14, 95\% CI [0.45, 1.83]). There was no difference in FFR amplitudes between high (M = 1.46 dB, 95\% CI [-0.51, 3.43]) and low performers (M = 3.34 dB, 95\% CI [1.17, 5.51]) to the 130 Hz tone (difference = 1.88 dB, SE = 1.08, \textit{t}(59.4) = 1.75, \textit{p} = .09, \textit{d} =0.60, 95\% CI [0.09, 1.29]).

The interaction between target frequency and task performance also reached significance. High task performers had lower FFR amplitudes across all target tones. This effect accounted for 0.8\% of the variance in FFR amplitudes (\textit{R}²$\beta$* = .008, 95\% CI [.001, .039]). When the target was 130 Hz, low performers (M = 4.63 dB, 95\% CI [2.46, 6.80]) had a higher FFR amplitude than high task performers (M = 2.42 dB, 95\% CI [0.45, 4.39]; difference = 2.22 dB, SE = 1.08, \textit{t}(59.4) = 2.06, \textit{p} = .044, \textit{d} = 0.70, 95\% CI [0.02, 1.39]). Similarly, when the target was 200 Hz, low performers (M = 5.47 dB, 95\% CI [3.30, 7.64]) had a higher FFR amplitude than high task performers (M = 2.23 dB, 95\% CI [0.26, 4.20], difference = 3.24 dB, SE = 1.08, \textit{t}(59.4) = 3.01, \textit{p} = .004, \textit{d} = 1.03, 95\% CI [0.34, 1.72]). Lastly, when the target was 280 Hz, low performers (M = 5.77, 95\% CI [3.60, 7.94]) had a higher FFR amplitude than high task performers (M = 3.00, 95\% CI [1.03, 4.97], difference = 2.77 dB, SE = 1,08, \textit{t}(59.4) = 2.58, \textit{p} = .013, \textit{d} = 0.88, 95\% CI [0.19, 1.57].

Importantly, the interaction between tone frequency and target frequency and the three-way interactions between stimulus, target, and task performance did not reach significance (both \textit{p} > .83). These interactions explained < 0.4\% of the variance. 



fixed effects of tone frequency (130, 200, 280 Hz), target frequency (130, 200, 280 Hz)

tested the effects of passive versus active listening on FFR amplitudes

Perfectly phrased — you’ve got a crossed/nested design that’s too complex for a single omnibus MLM to be clean and interpretable. You’re right: it usually makes sense to fit a set of models, each addressing a specific research question. Here’s a way to structure it:
We analyzed FFR amplitude with a set of focused mixed-effects models, each aligned to a specific estimand in our crossed/nested design.
Task × Tone (Model 1): Tests the within-subject effect of listening task and its interaction with tone across all trials, estimating the population-averaged task effect.
Target effects (Model 2, active-only): Tests whether responses differ for target vs. non-target tones when targets are defined (active blocks), where target is nested within tone.
Moderation by performance/ordering (Model 3): Tests whether accuracy group and block order moderate task and tone effects. (\\adding passive_order does not change the outcome)

Pre-specification: Identify primary outcomes (e.g., task main effect on FFR amplitude; task×tone) and mark Model 2–3 as secondary/moderation analyses.
Error control: Emphasize planned contrasts with effect sizes and CIs; avoid fishing across many unplanned interactions.
Robustness: State that you verified stability to (i) adding passive_order to Model 1, (ii) alternative random-effects (e.g., adding a slope for passive), and (iii) equalized trial counts.

4

\subsection*{Behavioral results by groups}
The top third of participants who had the highest accuracy scores across all three blocks were split into one group while the remaining participants were split into two groups based on the median difference between scores in the first and third blocks. Participants who showed a greater increase in accuracy from the start and end of the study were labeled ‘improvers’, while participants whose performance decreased or plateaued were called ‘decliners’ (Figure 4.5). Participants were surveyed for their musical background, and years spent playing their primary instrument was used as a measure of years of musical experience. Years of musical experience were significantly higher in consistent high performers compared to decliners (t(16) = 2.59, p < 0.05), while there was no difference in musical experience between the other groups. In comparison, improvers had lower Auditory N-Back scores than decliners (t(15) = 2.12, p < 0.1) and experts (t(15) = 1.93, p < 0.1) (Figure 4.6).

