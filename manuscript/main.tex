Participants found this task relatively easy, and a few participants were at ceiling. The overall mean accuracy was 91\% (SD = 6.4\%). However, performance varied significantly depending on which tone served as the target (F(2, 56) = 25.7, p < 0.01). Post hoc pairwise comparisons with Bonferroni correction revealed that accuracy was significantly lower for trials with the 200 Hz tone as the target (M = 85.1\%, SD = 10.6\%) compared to both the 130 Hz target (M = 94.6\%, SD = 5.2\%; mean difference = 9.5\%, p < 0.01) and the 280 Hz target (M = 92.9\%, SD = 6.2\%; mean difference = 7.8\%, p < 0.01). There was no significant difference in accuracy between the 130 Hz and 280 Hz targets (mean difference = 1.7\%, p > 0.1). 

To facilitate subsequent analyses, a median split was used to divide participants into high and low task performers (Mdn = 91.3\%) (Figure 4.1). Additionally, a simple linear regression was performed between years of musical practice and task accuracy, revealing a significant positive correlation (b = 0.6\%, p = 0.026). A simple linear regression was also performed between the d-prime scores for the auditory N-back (calculated as the difference between the z-score for hits, and the z-score for false alarms), and task accuracy, although this regression did not reach significance $\beta$ = 0.54\%, SE  = 0.24, \textit{t} = 2.24, \textit{p} = .031











\subsection*{Behavioral results by groups}
The top third of participants who had the highest accuracy scores across all three blocks were split into one group while the remaining participants were split into two groups based on the median difference between scores in the first and third blocks. Participants who showed a greater increase in accuracy from the start and end of the study were labeled ‘improvers’, while participants whose performance decreased or plateaued were called ‘decliners’ (Figure 4.5). Participants were surveyed for their musical background, and years spent playing their primary instrument was used as a measure of years of musical experience. Years of musical experience were significantly higher in consistent high performers compared to decliners (t(16) = 2.59, p < 0.05), while there was no difference in musical experience between the other groups. In comparison, improvers had lower Auditory N-Back scores than decliners (t(15) = 2.12, p < 0.1) and experts (t(15) = 1.93, p < 0.1) (Figure 4.6).
