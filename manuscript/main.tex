intro -- 'dimension selective attention tested with target freq'


Participants found this task relatively easy, with a few participants at performance ceiling. The overall mean accuracy was 91\% (SD = 6.4\%). Performance varied significantly depending on which tone served as the target (F(2, 46) = 19.23, p < 0.01). Post hoc pairwise comparisons with Bonferroni correction revealed that accuracy was significantly lower for trials with the 200 Hz tone as the target (M = 85.1\%, SD = 10.6\%) compared to both the 130 Hz target (M = 94.6\%, SD = 5.2\%; mean difference = 9.5\%, t(22) = 4.62, p < 0.01) and the 280 Hz target (M = 92.9\%, SD = 6.2\%; mean difference = 7.8\%, t (22)= 5.14, p < 0.01). There was no significant difference in accuracy between the 130 Hz and 280 Hz targets (mean difference = 1.7\%, t(22) = 1.23, p = 0.23). 

To facilitate subsequent analyses, a median split was used to divide participants into high and low task performers (Mdn = 91.3\%). Participants completed a musical experience questionnaire, and years spent playing their primary instrument was used as a measure of years of musical experience. A simple linear regression revealed a significant positive correlation between years of musical experience and task accuracy $\beta$ = 0.62\%, SE  = 0.03, \textit{t} = 2.33, \textit{p} = .029. A simple linear regression was also performed between the d-prime scores for the auditory N-back (calculated as the difference between the z-score for hits, and the z-score for false alarms), and task accuracy, although this regression did not reach significance $\beta$ = 5.92\%, SE  = 3.8, \textit{t} = 1.5, \textit{p} = .14.


\subsection*{FFR results}
Targeted mixed-effects models selected to align with each specific research questions were fit to the data. These models were chosen to avoid an over-parameterized omnibus model and provides clearer estimates for each hypothesis. All models were fit using restricted maximum likelihood (REML) in R (Bates et al., 2015, lme4; Kuznetsova et al., 2017, lmerTest; Lenth, 2025, emmeans) with random intercepts for participants. Effect sizes were calculated for Tukey corrected post hoc tests using Cohen’s d and are reported for planned comparisons and effects only. Model fit was assessed by computing the marginal and conditional$R^2$ values (Nakagawa \& Schielzeth, 2013, performance version 0.15.0.3). The first model included fixed effects for tone frequency (130, 200, 280 Hz) and listening condition (passive vs active) to address the effect of listening condition on FFR amplitudes. The second model addressed the effects of selective auditory attention to stimuli within an attended stream, with fixed effects for tone frequency (130, 200, 280 Hz) and target frequency (130, 200, 280 Hz). The last model tested the effects of accuracy and the interaction between accuracy and listening condition on the FFR using fixed effects for tone (130, 200, 280 Hz), condition (passive vs active) and accuracy (low vs high) (Figure 2).

The first model was fit with fixed effects for listening condition and tone frequency with random intercepts for participants to test for the effects of active listening on FFR amplitudes. The fixed effects of listening condition and tone frequency accounted for a large proportion of the variance (marginal $R^2$ = 26\%), and the full model including subject-level random effects explained 56\% (conditional $R^2$), reflecting strong individual-level predictability. The model had a significant intercept (b = 1.95, p < .001) indicating that the reference condition (active task, 130 Hz tone) had a mean FFR amplitude significantly above zero. FFR amplitudes were higher in during passive than active listening (b = 2.23, SE = 0.63, t(259) = 3.56, p < .001). A post hoc comparison to compute standardized effect size indicated that FFR amplitudes were significantly greater during passive than active listening (3.24 µV (SE = 0.36, 95\% CI [2.53, 3.95]), t(259)=8.94, p < .001) with a large standardized effect size (Cohen’s d = 1.22, 95\% CI [0.92, 1.52]). Amplitudes were also greater for the 200 Hz than 130 Hz tones (b=2.55, SE = 0.44, t(259) = 5.76, p < .001, mean difference = 3.52, SE = 0.44, t(259) = 7.94, p < .001) but did not different between the 280 Hz and 130 Hz tones (p = 0.40). A significant interaction between listening condition and tone frequency indicated that the increase in FFR amplitudes to the 200 Hz tone relative to the other tones was greater under passive listening (b = 1.93, SE = 0.89, t(259) = 2.17, p = .031). 

The second model with fixed effects for tone and target frequency revealed no effects dimension selective auditory attention on FFR amplitudes. Model fit indices suggest that tone and target frequency account only for 17\% of the variance in the model (marginal $R^2$), while the full model with subject-level random effects explained 47\% of the variance. This model also had a significant intercept (b = 2.06, p = .002), indicating that the FFR amplitudes for the reference condition (tone = 130 Hz, target = 130 Hz) was significantly above zero. Importantly, there were no significant main effects of target tone (ps > .60) and no significant interactions between target and tone frequency (all ps > .067). As with model 1, there was a significant main effect of tone frequency, FFR amplitudes were significantly higher to the 200 Hz than 130 Hz tone (b = 2.85, SE = 0.72, t(184) = 3.98, p < .001, post hoc effect size Cohen's d = 1.03, 95\% CI [0.67, 1.38]). Post hoc pairwise comparisons also reveal that FFR amplitudes were significantly higher to the 200 than 280 Hz tone (mean difference = 2.93, SE = 0.42, t(184) = 7.05, p < .001, d = 1.18, 95\% CI [0.82, 1.53]). 

The last model was fit with with fixed effects for accuracy, listening condition, and tone frequency to test for the effects of active listening on FFR amplitudes in experts and non-experts. The marginal $R^2$ indicated that the fixed effects accounted for a large portion of the variance in the model (marginal $R^2$ = 58\%) and the full model accounts for 33\% of the variance. This model revealed a significant interaction between tone frequency and accuracy with a greater FFR amplitude in low accuracy performances during the 280 Hz tone compared to the reference condition (b = 2.14, SE = 0.87, t(254.0) = 2.46, p = .015). Post hoc comparisons with Tukey corrections reveal lower FFR amplitudes in high task performers for both 130 Hz (mean difference = 2.16 uV, SE = 1.04, t(39.9) = 2.08, p = .045) and 200 Hz tones (mean difference = 3.19, SE = 1.04, t(39.9) = 3.06, p = .004). Notably, the intercept for this model was not significant, indicating that the FFR amplitudes for the reference condition (tone = 130 Hz, high accuracy, active listening) were not significantly above zero. The main effect of accuracy approached but did not reach significance (b = 1.92, p = .073). The interactions between accuracy and listening condition were also not significant (ps > .29). As with the other models, there was a main effect of tone frequency (b = 2.42, SE = 0.59, t(254.0) = 4.10, p < .001) and listening condition (b = 2.01, SE = 0.83, t(254.0) = 2.41, p = -.017), a useful robustness check for the effects of listening condition. No other interactions reached significance (ps > 0.33).

\subsection*{PLV}

The significance of observed PLVs were computed by comparing the observed group medians to the computed null distribution. The null distributed was computed as per Zhu and colleagues (2013) computing the mean of randomly drawn phase values 1000 times (Mdn = .042, 95\% CI [0.041, 0.045], \textit{B} = 1000). Since a majority of the observed PLV values were not normally distributed, their statistical significance was calculated as the proportion of values in the null distribution greater than the median of the observed PLVs. Mann-Whitney U tests were computed to compare the PLVs of high and low task performers. Wilcoxon signed-rank tests for between-subjects tests were used to compare the PLVs during active and passive listening. Cohen's r was computed for each test to assess effect sizes (Figure 3).

PLVs for both high and low performers exceeded the bootstrapped null at almost all tones (130 Hz: low accuracy Mdn = 0.103, 95\% CI [0.080, 0.147], p = 0.014; 200 Hz: low accuracy Mdn = 0.262, 95\% CI [0.118, 0.285], p < 0.001; high accuracy Mdn = 0.229, 95\% CI [0.099, 0.248], p < .001; 280 Hz low accuracy Mdn = 0.117, 95\% CI [0.065, 0.151], p = 0.001; high accuracy Mdn = 0.104, 95\% CI [0.065, 0.124], p = 0.013). Only the PLVs for high accuracy participants under the 130 Hz tone were not greater than null (Mdn = 0.078, 95\% CI [0.065, 0.086], p = 0.087). Moreover, PLVs for high and low accuracy participants were not significantly different under the 200 Hz (U = 63.0, p = .28, r = -.22) and 280 Hz tones (U = 73.0, p = .57, r = -.12). However, PLVs for high accuracy participants were lower than PLVs for low accuracy participants under the 130 Hz tone (U = 43.0, p = .035, r = -0.42).

PLVs were significantly greater than null under both active and passive listening conditions at all tones (130 Hz: active Mdn = 0.090, 95\% CI [0.075, 0.110], p = 0.037; passive Mdn = 0.090, 95\% CI [0.079, 0.100], p = 0.037; 200 Hz: active Mdn = 0.243, 95\% CI [0.122, 0.262], p < 0.001; passive Mdn = 0.219, 95\% CI [0.134, 0.282], p < 0.001; 280 Hz active = 0.098, 95\% CI [0.079, 0.122], p = .019; passive Mdn = 0.123, 95\% CI [0.088, 0.150], p < .001). Crucially, PLVs were not significantly different between the two listening conditions at all tones (130 Hz V = 155.0, p = .62, r = 0.029; 200 Hz V = 146.0, p = .47, r = -0.023; 280 Hz V = 99.0, p = 0.05, r = -0.30).


\subsection*{Behavioral performance by group}
To disentangle the relative effects of past experience and effort on the FFR, participants were split up into three groups: "experts", "improvers" and "decliners". Participants who had the highest accuracy scores across all three blocks we labeled as 'experts', participants who showed the greatest increase in accuracy from the start and end of the study among the remaining participants were labeled ‘improvers’, and participants whose performance decreased or plateaued throughout the experiment were labeled 'decliners'. Thresholds were adjusted such that each group had an equal number of participants (n = 8). Importantly, improvers and decliners had similar accuracy scores for the first block of the experiment (t(14) = 1.96, p = 0.07). Additionally, years of musical experience were significantly higher in experts compared to decliners (t(14) = 3.53, p = .003), while there was no difference in musical experience between the experts and improvers, or improvers and decliners (ps > 0.12). The auditory working memory of the three groups were not significantly different (ps < .090).







