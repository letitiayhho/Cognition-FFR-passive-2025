\documentclass{article}
\usepackage{graphicx} % Required for inserting images
\usepackage[style=apa, backend=biber]{biblatex}
\addbibresource{references.bib}  % your .bib file
\usepackage[utf8]{inputenc}
\usepackage[T1]{fontenc}
\usepackage{lineno}
\usepackage{authblk} % for author/affiliation formatting
\linenumbers 


\title{Expertise reduces subcortical auditory responses under task demands: Evidence for context-dependent neural efficiency}

\author[1]{Letitia Ho}
\author[1]{TBD}
\author[2]{Howard Nusbaum}

\affil[1]{Department of Psychology, University of Chicago, Chicago, IL, USA}

\affil[*]{Corresponding author: letitiayhho@uchicago.edu}

\date{} % leave date blank

\begin{document}

\maketitle

\section*{Abstract}

Auditory expertise has long been associated with larger frequency-following responses (FFRs), leading to the assumption that greater response amplitude reflects superior encoding fidelity. However, most prior studies have measured FFRs under passive listening conditions, leaving it unclear how expertise shapes neural encoding during active, attentionally demanding tasks. Here, we compared FFRs and event-related potentials (ERPs) in expert and non-expert listeners while participants selectively attended to target tones varying in frequency (130, 200, 280 Hz). Task performance was used as a proxy for expertise, and both amplitude and phase-locking value (PLV) of the FFR were assessed to distinguish neural population size from temporal precision. Mixed-effects models revealed that high performers exhibited significantly lower FFR amplitudes to 200 and 280 Hz tones, without reductions in PLV, suggesting more efficient recruitment of neural populations during active listening. FFR amplitudes were not modulated by target frequency, and no meaningful interactions emerged, indicating limited attentional effects on subcortical encoding. By contrast, robust effects of tone, target, and task performance were observed for the P2, consistent with strong top-down modulation at later cortical stages of auditory processing. Together, these findings challenge the assumption that auditory expertise uniformly increases neural gain. Instead, they support a framework in which expertise promotes efficiency and selectivity in early auditory encoding during active listening, while immediate task demands shape later cortical responses. Our results challenge common assumptions about FFR amplitude and highlight the need for a better understanding of how experience and attention interact to modulate different stages of auditory processing.

\section*{Introduction}

Experience and expertise shape auditory processing, with musicians and bilinguals often showing enhanced neural encoding of periodic acoustic structure. Much of this evidence comes from frequency-following responses (FFRs), which capture the early phase-locked encoding of sound. Under passive listening conditions, experts typically exhibit larger FFR amplitudes and greater phase consistency, leading to the assumption that greater FFR amplitudes necessarily reflect better encoding fidelity. However, it remains unclear whether the enhanced phase-locked response observed in experts reflects a fixed neural gain or a more flexible, context-dependent allocation of auditory resources.

Prior work has consistently found that auditory experts show enlarged FFRs in passive listening. Musicians, for example, often exhibit higher signal-to-noise ratios, more robust encoding of pitch, and greater resistance to noise masking in their FFRs \parencite{Parbery_Clark_2009, Strait_2009, Strait_2013, Wong_2007}. Similar enhancements also appear in bilinguals \parencite{Krizman_2012, Krizman_2014, Krizman_2014, Skoe_2017, Omote_2017} and tonal language speakers \parencite{Jeng_2011, Krishnan_2005}. These enhancements have been taken to support models of expertise as a general gain increase in the auditory system. Yet, because these studies record FFRs exclusively under passive conditions, it is uncertain whether the observed increases in phase-locking amplitude generalize to more naturalistic settings. Everyday listening often involves attention and the allocation of resources across competing sound streams. Whether expertise enhances encoding equally across passive and active conditions remains an open question.

The FFR has also been studied in relation to attention and task demands, but findings have been famously mixed. Some studies report that attention enhances FFR amplitude, while others report minimal or inconsistent effects compared to robust cortical modulations \parencite[e.g.][]{Galbraith_2003, Hairston_2013, Lehmann_2014, Galbraith_1993b, Hoormann_2000, Varghese_2015}. One possibility is that attentional effects on the FFR are small and difficult to detect. Another, which we test here, is that the effects of attention depend on expertise. Experts may not show the same attentional gain patterns as non-experts because their auditory systems are already tuned differently through long-term experience. The inconsistent attentional findings in the literature may thus reflect unmeasured heterogeneity in participants’ expertise and the contexts in which FFRs are elicited. 

Theoretical models of perceptual processing suggest that expertise may not always impart higher amplitude neural responses. Instead, models such as predictive coding \parencite{Rao_1999, Friston_2005} and biased competition \parencite{Chelazzi_1993, Li_1993} predict that experts may achieve better encoding with less cognitive resources by reducing redundant activity. Related work on dimension-selective attention \parencite{Holt_2018} further suggests that experts may learn to down-weight irrelevant or overlearned acoustic features, modulating their neural responses based on context. Together, these perspectives predict that expertise should not be associated with uniformly larger FFRs, but rather with context-dependent modulation of phase-locked responses.

We compared the FFRs of expert and non-expert listeners under active attention conditions requiring selective monitoring of sound features. FFRs were measured using a full EEG electrode array. Give that attention effects are mixed for the FFR but well established for the late, cortical responses indexed by the ERP, comparing FFRs and ERPs provides a way to pinpoint the locus of top-down attention effects on auditory processing. Task performance was used as a measure of expertise as it reflects both the accuracy of internal stimulus representations and is correlated with years of musical training. Additionally, measures of FFR amplitude were compared with phase-locking consistency. While these measures are often conflated as general indexes of FFR “strength”, amplitude reflects the overall power of phase-locked responses, whereas the phase-locking value \parencite[PLV; ][]{Lachaux_1999, Zhu_2013} specifically measures the temporal precision of neural firing across trials. Considering both is important because changes in amplitude may reflect either the size of the responding neural population or reduced phase consistency due to interference. Lastly, FFRs were evoked by single-polarity stimuli and referenced to the common average reference (CAR) to preserve contributions from both cortical and subcortical sources \parencite{Bidelman_2015b}.

\section*{Materials \& Methods}

A total of 45 participants were recruited for this study from the University of Chicago community (11 male, 1 non-binary, with ages ranging from 20-50 years old. All participants had normal speech and hearing as verified with an audioscope (Welch Allyn Audioscope 3). A further 3 participants were excluded from further analysis due to the high number of trials rejected following the PREP preprocessing pipeline (Bigdely-Shamlo et al., 2015), which utilized a peak-to-peak rejection threshold of 35 mV on the channel used to compute FFRs. The resulting dataset for the final analysis includes a total of 41 participants. This sample size was chosen to be approximately double that of prior studies of FFR expertise and attention \parencite[e.g., ][]{Sch_ller_2023, Price_2020, Lehmann_2014, Parbery_Clark_2009, Song_2008} in order to provide adequate sensitivity to detect medium-to-large between-subject effects.

Prior to their participation, participants signed an informed consent form approved by the Institutional Review Board at the University of Chicago. Participants were compensated with either cash at \$20/hr or with course credits. Participants completed a version of the MBQ (Gfeller, 2000) adapted for online use. As the MBQ does not yield a composite score, number of years of musical training as a measure of musical experience (Vanden Bosch der Nederlanden et al., 2018). 

\subsection*{Stimuli and task}
					
Stimuli were presented on a PC running Ubuntu Studio 20.04 using the Python package PsychoPy (Peirce et al., 2019). Stimuli were 3 pure sine-wave tones at 130, 200, 280 Hz played in a random order in sequences 30, 36, or 42 tones in length. Tones onsets and offsets were smoothly ramped with a 5 ms Hanning window. The tones were 300 ms in duration with a jittered interstimulus interval from 200-300 ms. Stimuli were presented at 65–70 dB SPL over insert earphones (Etymotic ER3C). At the start of each trial, one of the three tones was randomly selected to be the target tone and participants were allowed to listen to the target tone as many times as they wished. Participants were then asked to silently count the number of times the target tone played during the trial and report their answer at the end of the sequence. Sequences were pseudo-randomized such that the first tone was never a target tone, and no tone would play more than three consecutive times. To incentivize task compliance, subjects were given one ‘point’ if they answered correctly, or if their guess came within 2 tones of the actual target count. Each block ended when subjects scored 18 points or more than 22 trials were attempted. Subjects completed 5 blocks that lasted approximately 8-10 minutes each. This resulted in approximately 850-1000 sweeps per tone. Behavioral performance for each trial was calculated as 1- the ratio between the error and the actual number of targets. The overall accuracy for each participant was calculated as their mean accuracy across all trials. Participants with accuracies below the group median were labeled as low performers and participants with accuracies above were labeled as high performers.
					
\subsection*{Electrophysiological data acquisition and preprocessing}
					
Prior to EEG cap placement, participants were told what to expect from the task and EEG procedure. The circumference of each participant’s head was measured to fit the actiCAP slim EEG 64-channel net (actiCAP, Brain Products GmbH, Germany). Participants were seated in a bright, sound-attenuated, and electromagnetically shielded room and asked to attend to the tones while minimizing eye blinks and other movement artifacts. After the experiment, the exact electrode positions were recorded using CapTrak (Brain Products GmbH, Germany). EEG data was collected with a sampling rate of 5 kHz using an ac- tiCHamp Plus amplifier (Brain Products GmbH, Germany). Two electrodes (originally AF7 and AF8 in the actiCAP layout) were affixed under the outer canthi of the left and right eyes to monitor eye movements (i.e., to be used as EOG channels).		

The EEG data were preprocessed using custom Python scripts and the MNE-Python library version 1.5.0 (Gramfort et al., 2013). The eye electrodes (AF7 and AF8) were rereferenced to Fp1 and Fp2, respectively (i.e., below the eye referenced to above the con- tralateral eye), to create bipolar EOG channels. The automated PREP pipeline was used to notch filter the data at 60 Hz, identify and exclude bad channels using threshold-based artifact rejection, and re-reference the EEG data to the average of all electrodes. Detailed participant-level preprocessing reports containing details such as trial counts, removed Independent Components, interpolated electrodes, and data visualizations for quality check are available with our dataset on OpenNeuro. A timing test was run before the experiment and the measured lag in event times was 27.2 ms with a jitter of 0.73 ms.
					
\subsection*{FFR analysis}
					
The common average reference (CAR) was applied during PREP preprocessing. CAR was used to reduce the biasing of the recorded response toward peripheral generators, such as the cochlear microphonic or auditory nerve, and to avoid overestimating the absolute amplitude of the FFR signal (Bidelman et al., 2015). The EEG data were epoched from -200 ms to 400 ms relative to stimulus onset. These epochs were then time-shifted by -27.2 ms to correct for the delay in auditory stimulus delivery, based on the discrepancy between event marker timestamps and the actual onset of the stimulus signal as recorded by the StimTrak system. The epoched signals were then bandpass filtered from 50 to 300 Hz using a finite impulse response (FIR) filter with a lower and upper transition bandwidth of 12.5 Hz and 75 Hz, respectively, and a filter length of 6.6 times the lower transition bandwidth. To estimate signal power at the stimulus frequency for each participant, trials were averaged in the time domain for each tone and target condition. Power spectral densities were calculated separately for the 200 ms pre-stimulus baseline and the 200 ms post-stimulus window of these averaged signals. Power was then expressed in decibels (dB) by taking the logarithm of the ratio between post-stimulus and baseline power, multiplied by 10.

FFR amplitudes were analyzed using linear mixed-effects models fitted by restricted maximum likelihood (REML) with the lme4 package in R (version 2025.05.1; Bates et al., 2015). Tone frequency (130, 200, 280 Hz), target frequency (130, 200, 280 Hz), and task performance (high vs. low accuracy) were entered as fixed effects, with a random intercept for subject. Degrees of freedom were estimated using Satterthwaite’s approximation via lmerTest (Kuznetsova et al., 2017). Post hoc pairwise comparisons with Tukey adjustment and Cohen’s d effect sizes were computed using emmeans (Lenth, 2025). Overall model fit was summarized with marginal and conditional R² values, estimating variance explained by fixed effects alone and by the full model, respectively (Nakagawa \& Schielzeth, 2013), using the performance package. Semi-partial R² values for individual fixed effects were estimated with the Nakagawa–Schielzeth–Johnson method (Jaeger, 2025).

\subsection*{PLV analysis}
					
To assess the consistency of phase-locked neural responses to the stimulus, the phase-locking value (PLV) was calculated using the method described by Lachaux and colleagues (1999) and Zhu and colleagues (2013). FFR epochs were cropped to retain only the post-stimulus segment. PLVs were calculated for each participant and condition using a bootstrapping approach. For each iteration, 400 trials were randomly sampled, and the fast Fourier trans form (FFT) was applied to each epoch. Phase values at the stimulus frequency were then extracted by taking the angle of the FFT’s complex output. The PLV for each bootstrap draw was calculated by averaging the complex exponentials of these phase values across the 400 trials. This procedure was repeated 1000 times to obtain a stable estimate of PLV for each stimulus condition and participant. To assess significance, a null distribution was generated by computing PLVs from samples of 400 randomly drawn phases 1000 times. As a majority of the observed PLV values were not normally distributed, their statistical significance was calculated as the proportion of values in the null distribution greater than the median of the observed PLVs. PLVs of high and low task performers were compared using the Mann-Whitney U tests and Cohen’s r was calculated to measure effect size.
					
\subsection*{ERP analysis}
					
EEG data preprocessed following the electrophysiological data preprocessing steps described in Experiment 1 were used. The PREP preprocessed data was bandpass filtered from 0.1 to 40 Hz using a FIR filter with a lower and higher transition bandwidth of 0.1 Hz and 10 Hz and a filter length of 6.6 times the lower transition bandwidth. The data were then epoched from -200 ms to 500 ms relative to stimulus presentation onset and then downsampled to 1 kHz. To remove eye and muscle artifacts, independent component analysis (ICA) was applied to the preprocessed EEG data to decompose the signal into 15 independent components (ICs). To identify components reflecting EOG artifacts, each component’s time course was correlated with the EOG channels and the resulting correlation coefficients were z-scored. Any IC with a z-score of greater than 1.96 was removed from the EEG signal. Trials were baseline corrected using the 200 ms window prior to stimulus onset. Trials were then automatically rejected using the Autoreject package based on their peak-to-peak amplitudes (Jas et al., 2017). Finally, epochs were averaged in the time domain by condition and target tone to isolate the evoked potential.

Amplitudes for ERP components were calculated by taking the mean amplitude within a time window determined by visual inspection of the time-domain signals and from typical analysis windows used in ERP research (Key et al., 2005). The window for the N1 component was 100-150 ms, the window for the P2 component was 150-230 ms (Coch et al., 2005), the window for the N2 component was 250-360 ms (Folstein et al., 2007; Nieuwenhuis et al., 2003) and the window for the P3b component was 380-440 (Lange, 2009; Dien et al., 2010, Polich, 2007). Latencies for each peak were calculated by identifying the time point corresponding to the maximum or the minimum value of the peak in each window. For the small peak or inflection point in the P3b window, the latency of the peak was identified as the time point of the zero-crossing of the first derivative of the discrete time series. Additionally, the time window for the P3b component was shifted to 350-420 ms in Experiment 2 to account for the earlier peak latency.

Mixed-effects models similar to those computed for FFR analyses were fit for data on each ERP peak. The models were fit by restricted maximum likelihood (REML) with tone frequency (130, 200, 280 Hz), target frequency (130, 200, 280 Hz), and task performance (high vs. low accuracy) included as fixed effects with a random intercept for subject, and degrees of freedom were estimated using Satterthwaite’s approximation (Kuznetsova et al., 2017, lmerTest). Post hoc Tukey-adjusted comparisons and Cohen’s d effect sizes were computed with emmeans (Lenth, 2025). Model fit was summarized with marginal and conditional R² values (Nakagawa \& Schielzeth, 2013; performance), and semi-partial R² values for fixed effects were estimated using the Nakagawa–Schielzeth–Johnson method (Jaeger, 2025).




\section*{Results}
\subsection*{Behavioral results}

Participants generally performed well at the task with a few participants at ceiling. Overall mean task accuracy was 91\% (SD = 6.3\%). Task accuracy varied significantly with target frequency (\textit{F}(2, 88) = 58.7, \textit{p} < .001). Post hoc pairwise comparisons with Bonferroni correction applied indicates that accuracies were significantly lower for the 200 Hz target (M = 85.6\%, SD = 9.8\%) than the 130 Hz target (M = 95.2\%, SD = 4.6\%; mean difference = 9.62\%, \textit{p} < .001) and the 280 Hz target (M = 93.1\%, SD = 6.5\%; mean difference = 7.52\%, \textit{p} < .001). Accuracies were not significantly different between the 130 Hz and 280 Hz targets (mean difference = 2.09\%, \textit{p} = .36). A median split was taken to separate participants into high and low task performers (Mdn = 92.3\%) (Figure 1). An Ordinary Least Squares regression was run between years of musical experience and task accuracy. The model was weakly statistically significant \textit{R}² = .13, \textit{F}(1, 39) = 5.03, \textit{p} = .031, with years of musical experience significantly predicting scores $\beta$ = 0.54\%, SE  = 0.24, \textit{t} = 2.24, \textit{p} = .031.

\begin{figure}
    \centering
    \includegraphics[width=1\linewidth]{fig1-behavioral.png}
    \caption{Task accuracy was high across all three target conditions. Participants subjectively reported that the middle tone was more difficult, and this is reflected in the accuracy rates.}
    \label{Figure 1}
\end{figure}

\subsection*{FFR results}
A linear mixed-effects model was fit to the data using restricted maximum likelihood (REML) in R (version 2025.05.1; Bates et al., 2015, lme4; Kuznetsova et al., 2017, lmerTest; Lenth, 2025, emmeans). The model included fixed effects of tone frequency (130, 200, 280 Hz), target frequency (130, 200, 280 Hz), and task performance (high vs. low accuracy) on FFR amplitude, and a random intercept for subject, with degrees of freedom estimated via the Satterthwaite approximation. Effect sizes for post hoc tests measured using Cohen’s d are reported for the main planned comparisons and effects only (Figures 2 and 3). 

Model fit indices showed a marginal \textit{R}² = .16, representing variance explained by fixed effects, and a conditional \textit{R}² = .55, representing variance explained by both fixed and random effects (Nakagawa \& Schielzeth, 2013, performance version 0.15.0.3). The model indicated substantial between-subject variability (random intercept variance = 8.77, SD = 2.96), with residual variance of 9.89 (SD = 3.15). The intercept of the model was significant, \textit{b} = 1.90, SE = 0.90, \textit{t}(130.06) = 2.11, \textit{p} = .037, showing that the reference condition (stimulus = 130 Hz, target = 130 Hz, high performers) elicited reliably above-zero FFR amplitudes.

The model revealed no significant main effects, including the main effects of stimulus (200 Hz: \textit{b} = 0.61, SE = 0.93, \textit{t}(320) = 0.66, \textit{p} = .51; 280 Hz: \textit{b} = 0.94, SE = 0.93, \textit{t}(320) = 1.01, \textit{p} = .31), target (200 Hz: \textit{b} = -0.75, SE = 0.93, \textit{t}(320) = -0.81, \textit{p} = .42; 280 Hz: \textit{b} = -0.57, SE = 0.93, \textit{t}(320) = -0.62, \textit{p} = .54), or task performance (low vs. high: \textit{b} = -0.01, SE = 1.34, \textit{t}(130.06) = -0.01, \textit{p} = .99). All main effects explained \textit{R}²$\beta \leq$* .1\% of the variance.

The model indicated a significant interaction between tone frequency and task performance. This interaction explains 1.2\% of the total variance in FFR amplitudes (\textit{R}²$\beta$* = .012, 95\% CI [.001, .046]). FFR amplitudes were significantly higher in low task performers compared to high task performers for the higher frequency tones. When the stimulus frequency was 200 Hz, the low performers (M = 5.76 dB, 95\% CI [3.59, 7.93]) showed higher FFR amplitudes relative to the high performers (M = 2.98 dB, 95\% CI [1.01, 4.96]; difference = -2.77 dB (SE = 1.08, \textit{t}(59.4) = 2.58, \textit{p} = .013, \textit{d} = -0.88, 95\% CI [0.19, 1.57]). Similarly, when the stimulus frequency was 280 Hz, the low performers (M = 6.77 dB, 95\% CI [4.60, 8.94]) showed higher FFR amplitudes relative to the high performers (M = 3.20 dB, 95\% CI [1.23, 5.17]; difference = 3.57 dB, SE = 1.08, \textit{t}(59.4) = 3.32, \textit{p} = .002, \textit{d} = 1.14, 95\% CI [0.45, 1.83]). There was no difference in FFR amplitudes between high (M = 1.46 dB, 95\% CI [-0.51, 3.43]) and low performers (M = 3.34 dB, 95\% CI [1.17, 5.51]) to the 130 Hz tone (difference = 1.88 dB, SE = 1.08, \textit{t}(59.4) = 1.75, \textit{p} = .09, \textit{d} =0.60, 95\% CI [0.09, 1.29]).

The interaction between target frequency and task performance also reached significance. High task performers had lower FFR amplitudes across all target tones. This effect accounted for 0.8\% of the variance in FFR amplitudes (\textit{R}²$\beta$* = .008, 95\% CI [.001, .039]). When the target was 130 Hz, low performers (M = 4.63 dB, 95\% CI [2.46, 6.80]) had a higher FFR amplitude than high task performers (M = 2.42 dB, 95\% CI [0.45, 4.39]; difference = 2.22 dB, SE = 1.08, \textit{t}(59.4) = 2.06, \textit{p} = .044, \textit{d} = 0.70, 95\% CI [0.02, 1.39]). Similarly, when the target was 200 Hz, low performers (M = 5.47 dB, 95\% CI [3.30, 7.64]) had a higher FFR amplitude than high task performers (M = 2.23 dB, 95\% CI [0.26, 4.20], difference = 3.24 dB, SE = 1.08, \textit{t}(59.4) = 3.01, \textit{p} = .004, \textit{d} = 1.03, 95\% CI [0.34, 1.72]). Lastly, when the target was 280 Hz, low performers (M = 5.77, 95\% CI [3.60, 7.94]) had a higher FFR amplitude than high task performers (M = 3.00, 95\% CI [1.03, 4.97], difference = 2.77 dB, SE = 1,08, \textit{t}(59.4) = 2.58, \textit{p} = .013, \textit{d} = 0.88, 95\% CI [0.19, 1.57].

Importantly, the interaction between tone frequency and target frequency and the three-way interactions between stimulus, target, and task performance did not reach significance (both \textit{p} > .83). These interactions explained < 0.4\% of the variance. 


\begin{figure}
    \centering
    \includegraphics[width=0.75\linewidth]{fig2-FFR-time-domain.png}
    \caption{Grand average FFRs in the time domain for each tone with stimulus onset is at time 0. The transient brainstem response can be seen in the first 0-50 ms of the waveform, followed by the stable, phase-locked response. 
}
    \label{Figure 2}
\end{figure}

\begin{figure}
    \centering
    \includegraphics[width=0.75\linewidth]{fig3-FFR.png}
    \caption{FFR amplitudes to each tone. Each trace shows the magnitude spectrum of the evoked response to each tone relative to the baseline period, lines show means and the shaded regions represent the 95\% CI. (Top) There is significant power at the pitch of each tone and negligible power at all other frequencies. (Bottom) High task performers had much lower magnitude spectrum amplitudes than low task performers.
}
    \label{Figure 3}
\end{figure}

\subsection*{PLV results}
To verify whether the observed differences in FFR amplitude may be attributed to reduced phase locking consistency, PLV values were calculated for each tone for both low and high-performing participants. As described in the methods, PLV values were computed by bootstrapping values from samples of randomly drawn trials, while the null distribution was created by computing the mean of randomly drawn phase values 1000 times (Mdn = 0.044, IQR = 0.001, \textit{B} = 1000). Since a majority of the observed PLV values were not normally distributed, their statistical significance was calculated as the proportion of values in the null distribution greater than the median of the observed PLVs. To compare the PLVs of high and low task performers, Mann-Whitney U tests were used to compare group medians alongside Cohen’s r to measure effect size (Figure 4).

PLVs for all performance groups to all tones were significantly greater than the bootstrapped null. The median PLV to the 130 Hz tone for low performers (Mdn = 0.046, IQR = 0.005) with a BCa 95\% CI of [0.045, 0.048] was significantly greater than null (\textit{p} = .014, \textit{B} = 1000). Similarly, phase locking to the 130 Hz tone for high performers was Mdn = 0.049 (IQR = 0.005) with a BCa 95\% CI of [0.046, 0.050], which was significantly greater than null (\textit{p} < .001, \textit{B} = 1000). Phase locking to the 200 Hz tone for low performers (Mdn = 0.048, IQR = 0.011) with a BCa 95\% CI of [0.044, 0.052] and high performers (Mdn = 0.050, IQR = 0.012) with a BCa 95\% CI of [0.045, 0.057] were significantly greater than null (\textit{p} < .001, and \textit{p} <  .001 respectively, \textit{B} = 1000). Lastly, phase locking to the 280 Hz tone for low performers (Mdn = 0.051, IQR = 0.010) with a BCa 95\% CI of [0.046, 0.054] and high performers (Mdn = 0.047, IQR = 0.004) with a BCa 95\% CI of [0.045, 0.049] were significantly greater than null (\textit{p} < .001, and \textit{p} < .001 respectively, \textit{B} = 1000).

Phase locking was not significantly different between high and low task performers for the 200 and 280 Hz stimulus tones. The Mann-Whitney U tests between high and low task performers to the 200 Hz tone yielded U = 243.0, \textit{p} = 0.27 with a small effect size of r = 0.10. The test between high and low task performers to the 280 Hz tone yielded U = 177.0, \textit{p} = 0.86 with a small effect size of r = -0.16. In contrast to the higher tones, the test between PLVs for high and low task performers to the 130 Hz tone suggests lower phase locking consistency in low task performers U = 289.0, \textit{p} = 0.04 with a medium effect size of r = 0.27.

\begin{figure}
    \centering
    \includegraphics[width=1\linewidth]{fig4-PLV.png}
    \caption{Consistency of phase locking to each tone by accuracy group. PLVs across all tones and groups were significantly greater than null, while PLVs between accuracy groups were not significantly different for any of the tones. Dotted lines represent the median value of each distribution.}
    \label{Figure 4}
\end{figure}

\subsection*{ERP results}
Linear mixed-effects models were fit to the mean amplitudes of each ERP with tone frequency (130, 200, 280 Hz), target frequency (130, 200, 280 Hz) and task performance (high vs low accuracy) as fixed effects with a random intercept for subject and degrees of freedom estimated via the Satterthwaite approximation. As with FFR amplitude, models of ERP amplitude were fit using restricted maximum likelihood (REML) in R. Effect sizes for post hoc tests measured using Cohen’s d are reported for the main planned comparisons and effects only. While models fit to the P2 and N2 showed significant main effects and interactions of tone frequency, target frequency, the models fit to the N1 and P3b peaks did not (Figure 5).

The model fit to the P2 peak showed significant main effects of tone frequency and target frequency and a significant interaction between the two variables. Model fit indices showed a marginal \textit{R}² = 0.13, representing variance explained by fixed effects, and a conditional \textit{R}² = 0.56 representing variance explained by both fixed and random effects. The model indicated substantial between-subject variability (random intercept variance = 0.25, SD = 0.50), with residual variance of 0.20 (SD = 0.45). The model revealed significant main effects of tone frequency (\textit{R}²$\beta$* = .016, 95\% CI [.002, .055]), with significantly lower P2 amplitudes to the 130 Hz  (M = 0.66 $\mu$V, 95\% CI [0.44, 0.89]) than the 280 Hz tone (M = 0.61 $\mu$V, 95\% CI [0.39, 0.84], b = 0.45, SE = 0.15, t(296) = 3.08, p = .002). The model also indicated a significant main effect of target frequency, \textit{R}²$\beta$* = .023, 95\% CI [.004, .066]; however, these differences were not significant in post hoc pairwise comparisons after Tukey correction (all p > .40). Lastly, there was a significant interaction between tone and target frequency (\textit{R}²$\beta$* = .048, 95\% CI [.020, .107]), with lower P2 amplitudes to target tones. P2 amplitudes were lower to the 130 Hz tone when the 130 Hz was the target (M = 0.36 $\mu$V, 95\% CI [0.06, 0.67]) than when the 200 Hz (M = 0.71 $\mu$V, 95\% CI [0.41, 1.02], difference = 0.35 $\mu$V, SE = 0.10, t(296) = 3.45, p = .002) or 280 Hz tones (M = 0.91 $\mu$V, 95\% CI [0.61, 1.22], difference = 0.55 $\mu$V, SE = 0.10, t(296) = 5.40, p < .001) were targets. Similarly, amplitudes were lower for the 280 Hz tone when the 280 Hz tone (M = 0.40 $\mu$V, 95\% CI [0.08, 0.70]) was the target compared to the when 130 Hz tone was the target (M = 0.88 $\mu$V, 95\% CI [0.57, 1.19], difference = 0.49, SE = 0.10, t(296) = 4.81, p < .001). There were no significant main effects or interactions with task performance.

The same model fit to the N2 peak showed no significant main effects. The only significant interaction was between tone and target frequency (\textit{R}²$\beta$* = .017, 95\% CI [.01, .06]); however, no meaningful pairwise comparisons were revealed by the post hoc test. The same mixed effects model fit to the N1 and P3b peaks showed no significant main effects or interactions (all ps > .078). The model coefficient for the interaction between 280 Hz tone and 200 Hz target for the P3b model approached significance, with post hoc pairwise comparisons suggesting that at 200 Hz, responses to target 130 Hz were significantly smaller than those to target 200 Hz (difference = –0.43 $\mu$V, SE = 0.12, t(296) = –3.48, p = .002) and significantly smaller than those to target 280 Hz (difference = –0.54 $\mu$V, SE = 0.12, t(296) = –4.36, p < .001). 

The effects of target frequency and task performance on the latency of each component were more limited than the amplitude effects. Significant interactions between tone and target, indicating an effect on ERP latencies when a tone is a target tone, were only found for the N2 component (\textit{F}4, 154) = 5.74, p < .001). There was a significant influence of task performance on the latency of the N1 (\textit{F}1, 39) = 15.62, p < .001) and P2 components (\textit{F}1, 39) = 17.23, p < .001). There were no effects of task performance or target frequency on the P3b component.


\begin{figure}
    \centering
    \includegraphics[width=1\linewidth]{fig5-ERP.png}
    \caption{Changes in ERP latencies and amplitudes in response to target tones, particularly in the P2 component and the P3b (top). ERP changes to target tones are more pronounced in high-accuracy participants (bottom). Lines represent the mean evoked potential across all participants and the shaded region represents the 95\% CI.}
    \label{Figure 5}
\end{figure}

\section*{Discussion}

Much of the existing literature equates auditory expertise with larger FFR amplitudes and, in corollary, assumes increased FFR amplitude as a marker of better auditory encoding. The findings of the present study challenge these assumptions. Under active attention, high task performers exhibited lower FFR amplitudes to the 200 and 280 Hz tones, even though their responses did not show reduced temporal precision as measured by phase-locking value (PLV). Instead, the decrease in amplitude occurred in conjunction with no changes in phase-locking consistency, suggesting that expertise does not simply amplify neural gain, but rather sharpens selectivity in the recruited neural population during active listening. While lack of interactions between tone and target suggest limited target effects on the FFR, at the cortical level, we observed robust target-related modulations of the P2, consistent with well-established attention effects on later, integrative processes. Taken together, these results encourage auditory expertise not to be understood as a uniform increase in response magnitude across levels of the auditory system, but as a context-dependent reallocation of resources that promotes efficiency and selective processing.

Studies on the FFR in experts typically report increased FFR amplitudes and consistency with expertise \parencite{Strait_2009, Strait_2013, Wong_2007}. For example, Parbery-Clark and colleagues \parencite*{Parbery_Clark_2009} compared FFRs to speech and speech in noise in musicians and non-musicians. They found that musicians had lower latency FFRs, greater power at the stimulus harmonics, and higher correlations between both their FFRs and the stimulus signals, and between their FFRs to speech and speech in noise. Similarly, Musacchia and colleagues \parencite*{Musacchia_2008} reported both earlier latency and higher amplitude FFRs in musicians compared to non-musicians to speech and music tokens. Comparable changes in early auditory responses have been recorded in tonal language speakers \parencite{Jeng_2011, Krishnan_2005} and bilinguals \parencite{Krizman_2012, Krizman_2014, Krizman_2015, Skoe_2017, Omote_2017}. However, these studies typically record FFRs under passive listening conditions. Even studies investigating the effects of training on early auditory encoding record FFRs separately from the tasks used to measure learning and perceptual improvement \parencite{Russo_2005}. Song and colleagues taught participants to associate mandarin pseudowords with images of common objects and then recorded FFRs to mandarin tones in a separate task, while participants watched a silent video \parencite*{Song_2008}. These results have led the field to equate higher amplitude FFR responses with better encoding fidelity.

The reduction in FFR amplitude observed in high task performers, alongside no change in PLV, suggests that expertise alters the size of the recruited neural population rather than the temporal precision of phase locking. Although amplitude and PLV are often treated as general indices of FFR “strength,” the two measures capture distinct properties of the FFR and may vary independently. Amplitude reflects the overall power of phase-locked activity and is often taken as a proxy for the number of neurons contributing to the response \parencite{Luck_2014, Mitzdorf_1985}. By contrast, PLV is derived by extracting phase information at the frequency of interest from each trial’s Fourier transform, and then taking the length of the average vector across trials as an index of the consistency of phase alignment \parencite{Zhu_2013, Lachaux_1999}. This makes PLV a robust measure of temporal synchronization that is independent of amplitude. Increases in FFR amplitude without corresponding changes in PLV therefore reflect a change in the recruitment of neural populations rather than greater temporal fidelity. In the present data, experts appear to recruit fewer neurons during active listening while maintaining precise phase locking, pointing to a more efficient allocation of resources in early auditory encoding.

This increase in encoding efficiency is consistent models of perception that frame expertise not as a non-selective, context-independent amplification of responses, but as the ability to selectively reduce of redundant neural activity under particular conditions. Predictive coding models propose that the brain suppresses responses to predictable or repeated input, thereby reducing unnecessary neural activity while preserving precision for behaviorally relevant features \parencite{Rao_1999, Friston_2005}. From this perspective, the reduced FFR amplitudes in high performers may reflect the ability to efficiently filter out redundant information during active listening and conserve neural resources. Relatedly, biased competition models \parencite{Chelazzi_1993, Desimone_1995} argue that perception is the outcome of competition for limited neural processing capacity. Experts may be better at flexibly allocating these resources by sharpening their response without an accompanying net increase in the overall amplitude of neural activity. Thus, rather than interpreting smaller FFR amplitudes as reduced encoding fidelity, our results are consistent with models in which expertise enhances neural efficiency and selectivity, enabling more adaptive responses under task oriented listening conditions.

The contrast between minimal attentional effects on the FFR and the robust effects of target frequency on the P2 suggests that experience and attention shape different stages of auditory processing. Long-term expertise appears to sharpen selectivity in the early, phase-locked response, while short-term task demands exert limited influence at this stage. In contrast, task performance had no significance association with changes in ERPs, but P2 amplitudes were significantly lower to target tones. The P2 has been linked to stimulus evaluation and categorization \Parencite{Luck_Hillyard_1994, Naatanen_Winkler_1999}. The strong target-related modulations of the P2 observed here indicate that changes in top-down attention with task demands may primarily influence auditory processing at late, cortical stages, stages which reflect decision making or integrative processes. This division is further supported by the limited cortical phase locking at higher frequencies reported in other studies \parencite[>200 Hz; ][]{Tichko_2017, Coffey_2016}. The fact that PLV differences were recorded at 130 Hz but not at 200 or 280 Hz supports the idea that attention effects on phase-locked encoding likely reflect modulations in cortical rather than subcortical sources \parencite{Hartmann_Weisz_2019, Holmes_2017, Forte_2017}. In conjunction, these findings suggest that experience tunes subcortical selectivity, conserving resources during early encoding, while attention flexibly modulates later cortical responses to support stimulus categorization and decision-making.

The results presented here challenge the assumption that expertise and experience increase the gain of early auditory responses across all conditions, and therefore that greater FFR amplitudes necessarily indicate better auditory encoding. The results point instead toward a model of auditory encoding in which long-term experience and short-term attention make complementary contributions. Expertise appears to enable the sharpening of early, subcortical phase-locked responses by reducing redundancy and increasing selectivity, consistent with efficiency-based accounts of neural plasticity. In contrast, attention operates most strongly at later cortical stages, where signals are integrated, evaluated, and mapped onto task demands. This separation helps reconcile prior inconsistencies in the FFR literature. Studies that reported minimal attentional effects on the FFR may in fact have been capturing a boundary condition where task demands exert little influence on early encoding. At the same time, robust attentional modulations of cortical ERPs underscore that top-down mechanisms influence only late, downstream auditory processes, with limited influence on early subcortical responses. Together, this framework emphasizes that auditory expertise should not be understood as a general gain increase, but as a system-level reorganization in which subcortical selectivity is shaped by experience and late, integrative cortical processes are shaped by attention.

Several constraints of the present study should be acknowledged. First, task performance was used as a proxy for expertise, an approach that is not uncommon in perceptual and cognitive research \parencite[e.g., ][]{Luthra_2024, Gauthier_1998}. Nevertheless, task performance likely reflects a mixture of both long-term expertise and short-term effort, and future work should aim to disentangle these contributions through independent measures of training history, effort, and cognitive resources. Second, we employed single-polarity stimuli, which may limit comparisons with studies using the more common alternating-polarity stimuli. However, this choice enabled us to preserve both cortical and subcortical contributions to the phase-locked response \parencite{Tichko_2017, Bidelman_2015b}. Finally, the use of simple pure tones, while affording experimental control, restricts ecological validity. Extending this work to more complex and naturalistic stimuli, such as speech and music, will be crucial for understanding how expertise and attention interact in everyday listening.

The present findings challenge the longstanding assumption that auditory expertise manifests as uniformly larger neural responses. Instead, high task performers exhibited reduced FFR amplitudes under active listening without loss of phase precision, pointing to a flexible increase in efficiency and selectivity rather than overall gain. At the same time, robust target-related modulations of the P2 confirm that attention exerts its strongest influence at later cortical stages. Together, these results bridge the expertise and attention literatures by building a picture of how experience and attention respectively shape different stages of the auditory processing hierarchy. Our findings support a model of audition and auditory attention that emphasizes efficiency, selectivity, and context-dependent modulation across the auditory pathway.

\printbibliography

\end{document}
